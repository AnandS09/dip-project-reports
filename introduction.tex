\section{\textsc{introduction}}

In state of the art non-invasive surgery, many times a guide-wire is inserted in the vasculature system in order to keep from having to open up more parts of the body. 
Due to the non invasive nature, conventional imaging techniques fall short of providing the surgeons the complete picture of the location and movement of the intrusive devices within the body during the procedure. 
Therefore, Ultrasound and X-ray imaging is employed to track some devices inside the body. 
However, distinguishing between the internal soft and hard tissues from any surgical device in a X-ray or ultrasound image needs the perceptional skills of a trained radiologist. 
In a hectic surgery procedure, the difficulty of this multiplies manifold as the surgeon has to focus on the surgical procedure as well as tracking simultaneously.
An automated tracker therefore is of immense value, as it frees up the surgeon's mental effort from an rather mundane task and allows for better focus on the procedure.
This in turn reduces the probability of human error, making surgeries safer.

\textbf{Goal:} \textit{In this project we intend to develop a machine learning based real time tracker for objects moving in a video.}
%
We believe that this is an \textit{relevant} problem as, this is a pure image processing and perception application which will enable us to get a deeper understanding on the practical aspects of the concepts learnt in this course. 
We also believe that this is an \textit{important} problem as the final product will prove to be a valuable resource for improving non-invasive surgical procedures.




We first aim to demonstrate the ability of our algorithms in a general purpose video tracking data set like Tracking Net \cite{trackingnet} and then transfer the model to an actual medical imaging setting. 
The justification of this approach is the procuring medical image dataset can prove to be problematic. 

%The following section describes our project plan in detail.